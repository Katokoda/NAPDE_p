%%%%%%%%%%%%%%%%%%%%%%%%%%%%%%%%%%%%%%%%%%%
\documentclass[11pt, a4paper, twoside]{article}
\usepackage[utf8]{inputenc}
\usepackage[T1]{fontenc}
\usepackage[english]{babel}
\usepackage{lmodern}
\usepackage{geometry}
\geometry{hmargin=2cm,vmargin=1.5cm}
\usepackage{graphicx}

\usepackage{xcolor}
\usepackage{calrsfs} %% for nice calligraphic letters

\usepackage{dsfont}
\usepackage{amsmath}
\usepackage{amssymb}
\usepackage{amsthm}
\usepackage{enumitem}
\usepackage{parskip}
\usepackage{comment}

\usepackage{fancyvrb} % for verbatime for code

\usepackage{dsfont}
\usepackage{caption}
\captionsetup[figure]{labelfont=bf} % for title figures
\usepackage{hyperref} % to reference the figures
\usepackage{float} % to fix the figure placement because latex is a b*tch
%%%%%%%%%%%%%%%%%%%%%%%%%%%%%%%%%%%%%%%%%%%
\newcommand{\floor}[1]{\lfloor #1 \rfloor} % partie entière

\newcommand{\N}{\mathbb{N}} % nb naturels
\newcommand{\Z}{\mathbb{Z}} % entiers relatifs
\newcommand{\R}{\mathbb{R}} % c'est réel
\newcommand{\Q}{\mathbb{Q}} % nb rationnels
\renewcommand{\P}{\mathbb{P}} % nb premiers ou proba
\newcommand{\vp}{\varphi}
\newcommand{\eps}{\varepsilon}
\newcommand{\inv}{^{-1}}
\newcommand{\ind}{\mathds{1}} % indicatrice

\newcommand{\rest}[1]{\raise-.5ex\hbox{\ensuremath|}_{#1}} % restriction
\newcommand{\grad}{\nabla} % gradient
\renewcommand{\div}{\text{div}} % divergence

\newcommand{\Ltwo}{L^2(\Omega)}
\newcommand{\Linf}{L^\infty(\Omega)}
\newcommand{\Hun}{H^1(\Omega)}
\newcommand{\Hunz}{H^1_0(\Omega)}
\newcommand{\Cinfz}{C^\infty_0(\Omega)}
\newcommand{\intom}{\int_\Omega}
\newcommand{\intbd}{\oint_{\partial \Omega}}
\renewcommand{\d}{\text{d}}

\newcommand{\norm}[2]{|| #2||_{#1}}

\newcommand{\st}{such that }
\newcommand{\ie}{i.e. }
\newcommand{\wrt}{with respect to }
\newcommand{\sub}{\subseteq}

%%%%%%%%%%%%%%%%%%%%%%%%%%%%%%%%%%%%%%%%%%%

\begin{document}
\title{Project 3: A semilinear elliptic equation}
\author{}
\date{Spring 2023}
\maketitle 

%%%%%%%%%%%%%%%%%%%%%%%%%%%%%%%%%%%%%%%%%%%
\subsection*{Question 1}
%%%%%%%%%%%%%%%%%%%%%%%%%%%%%%%%%%%%%%%%%%%
We want to find the weak formulation to the problem of finding $u_{n+1}$ satisfying
$$\begin{cases}  - \Delta u_{n+1} + \alpha u_n^2 u_{n+1} = f & \text{in } \Omega,\\
u_{n+1} = 0 &\text{on } \partial\Omega,
\end{cases}$$
where $u_n$ is the current iterate, with $u_n = 0$ on $\partial\Omega$.\newline

We multiply by a test function $v\in \Hunz$ and integrate to obtain:
$$\intom fv = \intom \left( -\Delta u_{n+1} v + \alpha u_n^2 u_{n+1} v \right).$$
We use integration by parts to rewrite the right hand side:
\begin{align*}
 \intom (-\Delta u_{n+1} v)
 &= - \intom \div(\grad u_{n+1} v) + \intom \grad u_{n+1} \grad v
 	&\text{integration by part}
 \\%%%%%%%%%%%%%
 &= - \intbd v (\grad u_{n+1} \cdot \vec n) + \intom \grad u_{n+1} \grad v
 	&\text{by the divergence theorem}
 \\%%%%%%%%%%%%%
 &=  \intom \grad u_{n+1} \grad v
 	&\text{by } v\in\Hunz,
\end{align*}
where we denote $\vec n$ the outer normal vector on the boundary $\partial \Omega$.

The weak formulation is:
\begin{equation}
\label{weak_f} \tag{E1}
\intom
\grad u_{n+1} \grad v
+ 
\intom
\alpha u_n^2 u_{n+1} v
= \intom fv
\quad \forall v\in\Hunz.
\end{equation}

We have a stiffness integral, and a mass integral with the reaction term $\alpha u_n^2$, as well as a Poisson right hand side.

%%%%%%%%%%%%%%%%%%%%%%%%%%%%%%%%%%%%%%%%%%%
\subsection*{Question 2}
%%%%%%%%%%%%%%%%%%%%%%%%%%%%%%%%%%%%%%%%%%%
The code for the implementation of the fixed-point scheme is in the files \verb+integrate.py+ and \verb+Project_SEZAM.py+. ADD WHAT EACH FILE CONTAIN
For $\alpha=0.1$, the value $||u_n -u_{n+1}||_\infty$ drops below $10^{-6}$ after 11 iterations. We obtain the figure \ref{q2fig1} below, by running \verb+fixed_point_method(0.1, 10e-6, 500, 0.05)+.
\begin{figure}[H]
\centering
\includegraphics[scale = 0.7]{../Figures/fixed_sol_alpha0.1.png}
\caption{Solution of the fixed-point scheme for $\alpha=0.1$}
\label{q2fig1}
\end{figure}

We also show the convergence rate in the figure \ref{q2fig2}.
\begin{figure}[H]
\centering
\includegraphics[scale = 0.7]{../Figures/fixed_conv_alpha0.1.png}
\caption{Convergence of the fixed-point scheme for $\alpha=0.1$}
\label{q2fig2}
\end{figure}

%%%%%%%%%%%%%%%%%%%%%%%%%%%%%%%%%%%%%%%%%%%
 For $\alpha=2$, it drops below $10^{-6}$ after 360 iterations, which is much longer. We obtain the figures \ref{q2fig3} and \ref{q2fig4} by running \verb+fixed_point_method(2, 10e-6, 500, 0.05)+.
 
 \begin{figure}[H]
\centering
\includegraphics[scale = 0.7]{../Figures/fixed_sol_alpha2.png}
\caption{Solution of the fixed-point scheme for $\alpha=2$}
\label{q2fig3}
\end{figure}

\begin{figure}[H]
\centering
\includegraphics[scale = 0.7]{../Figures/fixed_conv_alpha2.png}
\caption{Convergence of the fixed-point scheme for $\alpha=2$}
\label{q2fig4}
\end{figure}

%%%%%%%%%%%%%%%%%%%%%%%%%%%%%%%%%%%%%%%%%%%

%%%%%%%%%%%%%%%%%%%%%%%%%%%%%%%%%%%%%%%%%%%
\subsection*{Question 3}
The code for the implementation of the Newton scheme is also in the files \verb+integrate.py+ and \verb+Project_SEZAM.py+. ADD PRECISE FUNCTIONS
At each iteration, we need to search $\partial u_n$ that satisfies 
$$\intom \grad \phi \cdot \grad \partial u_n + 3\alpha u_n^2 \phi \partial u_n \d\Omega = - \intom \grad \phi \cdot \grad u_n + \phi (\alpha u_n^3 - f) \d\Omega \quad \forall \phi\in \Hunz, $$
$$\partial u_n = 0 \quad \text{on } \partial\Omega.$$
Then, we seek the solution $u$ as the limit of the recursive sequence $u_{n+1} =u_n + \partial u_n$.
%%%%%%%%%%%%%%%%%%%%%%%%%%%%%%%%%%%%%%%%%%%

For $\alpha = 0.1$, we obtain $||u_n -u_{n+1}||_\infty < 10^{-6}$ after 5 iterations, the solution is shown in figure \ref{q3fig1} below.
\begin{figure}[H]
\centering
\includegraphics[scale = 0.7]{../Figures/newt_sol_alpha0.1.png}
\caption{Solution with the Newton scheme for $\alpha=0.1$}
\label{q3fig1}
\end{figure}
We see that we find the same solution as with the fixed-point method (see figure \ref{q2fig1}), and we just need 5 less iterations with the Newton scheme (which actually divide the number of iterations by 2 in this case).
The convergence evolution of the Newton scheme for $\alpha=0.1$ is in the figure \ref{q3fig2}:
\begin{figure}[H]
\centering
\includegraphics[scale = 0.7]{../Figures/newt_conv_alpha0.1.png}
\caption{Convergence of the Newton scheme for $\alpha=0.1$}
\label{q3fig2}
\end{figure}
We obtain those figures by running \verb+newton_method(0.1, 10e-6, 500, 0.05)+.
%%%%%%%%%%%%%%%%%%%%%%%%%%%%%%%%%%%%%%%%%%%
For $\alpha=2$, we converge in 7 iterations, which is considerably faster than with the fixed-point scheme (we needed 360 iterations to converge with this scheme).

We obtain the following results by running \verb+newton_method(2, 10e-6, 500, 0.05)+:

\begin{figure}[H]
\centering
\includegraphics[scale = 0.7]{../Figures/newt_sol_alpha2.png}
\caption{Solution of the Newton scheme for $\alpha=2$}
\label{q3fig3}
\end{figure}

\begin{figure}[H]
\centering
\includegraphics[scale = 0.7]{../Figures/newt_conv_alpha2.png}
\caption{Convergence of the Newton scheme for $\alpha=2$}
\label{q3fig4}
\end{figure}

We also see that the solutions are the same for both scheme (figure \ref{q3fig3} above is the same as figure \ref{q2fig3}).
%%%%%%%%%%%%%%%%%%%%%%%%%%%%%%%%%%%%%%%%%%%
For $\alpha=5$, we also converge in 7 iterations, and obtain the results shown in figures \ref{q3fig5} and \ref{q3fig6} below (by running \verb+newton_method(5, 10e-6, 500, 0.05)+).

\begin{figure}[H]
\centering
\includegraphics[scale = 0.7]{../Figures/newt_sol_alpha2.png}
\caption{Solution of the Newton scheme for $\alpha=2$}
\label{q3fig5}
\end{figure}

\begin{figure}[H]
\centering
\includegraphics[scale = 0.7]{../Figures/newt_conv_alpha2.png}
\caption{Convergence of the Newton scheme for $\alpha=2$}
\label{q3fig6}
\end{figure}

The Newton scheme fare much better than the fixed-point scheme. The number of iterations to obtain $||u_n -u_{n+1}||_\infty < 10^{-6}$ is considerably smaller. (We ran the scheme for $\alpha=100$, and found that it converges in 10 iterations.)

%%%%%%%%%%%%%%%%%%%%%%%%%%%%%%%%%%%%%%%%%%%

\end{document}
