%%%%%%%%%%%%%%%%%%%%%%%%%%%%%%%%%%%%%%%%%%%
\documentclass[11pt, a4paper, twoside]{article}
\usepackage[utf8]{inputenc}
\usepackage[T1]{fontenc}
\usepackage[english]{babel}
\usepackage{lmodern}
\usepackage{geometry}
\geometry{hmargin=2cm,vmargin=1.5cm}
\usepackage{graphicx}

\usepackage{xcolor}
\usepackage{calrsfs} %% for nice calligraphic letters

\usepackage{dsfont}
\usepackage{amsmath}
\usepackage{amssymb}
\usepackage{amsthm}
\usepackage{enumitem}
\usepackage{parskip}
\usepackage{comment}

%\usepackage{bbold} % fonction indicatrice avec 1
\usepackage{dsfont}
%%%%%%%%%%%%%%%%%%%%%%%%%%%%%%%%%%%%%%%%%%%
\newcommand{\floor}[1]{\lfloor #1 \rfloor} % partie entière

\newcommand{\N}{\mathbb{N}} % nb naturels
\newcommand{\Z}{\mathbb{Z}} % entiers relatifs
\newcommand{\R}{\mathbb{R}} % c'est réel
\newcommand{\Q}{\mathbb{Q}} % nb rationnels
\renewcommand{\P}{\mathbb{P}} % nb premiers ou proba
\newcommand{\vp}{\varphi}
\newcommand{\eps}{\varepsilon}
\newcommand{\inv}{^{-1}}
\newcommand{\ind}{\mathds{1}} % indicatrice

\newcommand{\rest}[1]{\raise-.5ex\hbox{\ensuremath|}_{#1}} % restriction
\newcommand{\grad}{\nabla} % gradient
\renewcommand{\div}{\text{div}} % divergence

\newcommand{\Ltwo}{L^2(\Omega)}
\newcommand{\Linf}{L^\infty(\Omega)}
\newcommand{\Hun}{H^1(\Omega)}
\newcommand{\Hunz}{H^1_0(\Omega)}
\newcommand{\Cinfz}{C^\infty_0(\Omega)}
\newcommand{\intom}{\int_\Omega}
\newcommand{\intbd}{\oint_{\partial \Omega}}

\newcommand{\norm}[2]{|| #2||_{#1}}

\newcommand{\st}{such that }
\newcommand{\ie}{i.e. }
\newcommand{\wrt}{with respect to }
\newcommand{\sub}{\subseteq}

%%%%%%%%%%%%%%%%%%%%%%%%%%%%%%%%%%%%%%%%%%%

\begin{document}
\title{Project 3: A semilinear elliptic equation}
\author{}
\date{Spring 2023}
\maketitle 

\subsection*{Question 1}
We want to find the weak formulation to the problem of finding $u_{n+1}$ satisfying
$$\begin{cases}  - \Delta u_{n+1} + \alpha u_n^2 u_{n+1} = f & \text{in } \Omega,\\
u_{n+1} = 0 &\text{on } \partial\Omega,
\end{cases}$$
where $u_n$ is the current iterate, with $u_n = 0$ on $\partial\Omega$.\newline

We multiply by a test function $v\in \Hunz$ and integrate to obtain:
$$\intom fv = \intom \left( -\Delta u_{n+1} v + \alpha u_n^2 u_{n+1} v \right).$$
We use integration by parts to rewrite the right hand side:
\begin{align*}
 \intom (-\Delta u_{n+1} v)
 &= - \intom \div(\grad u_{n+1} v) + \intom \grad u_{n+1} \grad v
 	&\text{integration by part}
 \\%%%%%%%%%%%%%
 &= - \intbd v (\grad u_{n+1} \cdot \vec n) + \intom \grad u_{n+1} \grad v
 	&\text{by the divergence theorem}
 \\%%%%%%%%%%%%%
 &=  \intom \grad u_{n+1} \grad v
 	&\text{by } v\in\Hunz,
\end{align*}
where we denote $\vec n$ the outer normal vector on the boundary $\partial \Omega$.

The weak formulation is:
\begin{equation}
\label{weak_f}
\intom
\grad u_{n+1} \grad v
+ 
\intom
\alpha u_n^2 u_{n+1} v
= \intom fv
\quad \forall v\in\Hunz.
\end{equation}

We have a stiffness integral, and a mass integral with the reaction term $\alpha u_n^2$, as well as a Poisson right hand side.

\subsection*{Question 2}
The code for the implementation of the fixed-point scheme is in the files ???
For $\alpha=0.1$, the value $||u_n -u_{n+1}||_\infty$ drops below $10^{-6}$ after 11 iterations. For $\alpha=2$, it drops below $10^{-6}$ after ?? iterations, which is much longer.

\subsection*{Question 3}


\end{document}
